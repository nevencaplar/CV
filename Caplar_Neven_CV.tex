% Jason R. Blevins - Curriculum Vitae
%
% Copyright (C) 2004-2016 Jason R. Blevins <jrblevin@sdf.org>
% http://jblevins.org/
%
% You may use use this document as a template to create your own CV
% and you may redistribute the source code freely.  No attribution is
% required in any resulting documents.  I do ask that you please leave
% this notice and the above URL in the source code if you choose to
% redistribute this file.

\documentclass[11pt,letterpaper]{article}

\usepackage{hyperref}
\usepackage{geometry}
\usepackage{enumitem}

% Fonts
\usepackage[T1]{fontenc}


% Set your name here
\def\name{\textbf{Neven Caplar}}

% The following metadata will show up in the PDF properties
\hypersetup{
  colorlinks = true,
  urlcolor = black,
  pdfauthor = {\name},
  pdfkeywords = {black hole-galaxy co-evolution, time-domain astronomy, AGN, CV},
  pdftitle = {\name: Curriculum Vitae},
  pdfsubject = {Curriculum Vitae},
  pdfpagemode = UseNone
}

\geometry{
  body={6.9in, 9.8in},
  left=0.7in,
  top=0.7in
}

% Customize page headers
\pagestyle{myheadings}
\markright{\name}
\thispagestyle{empty}

% Custom section fonts
\usepackage{sectsty}
\sectionfont{\rmfamily\bfseries\LARGE}
\subsectionfont{\rmfamily\bfseries\large}

% Other possible font commands include:
%\rmfamily is original
% \ttfamily for teletype,
% \sffamily for sans serif,
% \bfseries for bold,
% \scshape for small caps,
% \normalsize, \large, \Large, \LARGE sizes.

% Don't indent paragraphs.
\setlength\parindent{0em}

% Make lists without bullets and compact spacing
\renewenvironment{itemize}{
  \begin{list}{}{
    \setlength{\leftmargin}{1.5em}
    \setlength{\itemsep}{0.15em}
    \setlength{\parskip}{0pt}
    \setlength{\parsep}{0.25em}
  }
}{
  \end{list}
}
\setlist[enumerate]{itemsep=0.22em}

\begin{document}

% Place name at left
{\huge \name}

% Alternatively, print name centered and bold:
%\centerline{\huge \bf \name}

\bigskip

\begin{minipage}[t]{0.495\textwidth}
  Princeton University\\
  Astrophysical Department \\
  08540 - Princeton, NJ, USA	 \\
	4 Ivy Lane
\end{minipage}
\begin{minipage}[t]{0.495\textwidth}
  Phone: +1 609 787 8425 \\
  Web: \href{ www.ncaplar.com}{www.ncaplar.com} \\
  Email: \href{ncaplar@princeton.edu}{\nolinkurl{ncaplar@princeton.edu}} \\
  Email2: \href{caplarn@phys.ethz.ch}{\nolinkurl{caplarn@phys.ethz.ch}} \\

\end{minipage}


\section*{Work experience}

\begin{itemize}
  \item 2017 - , Associate Professional Specialist, Princeton University
\end{itemize}  


\section*{Education}

\begin{itemize}
  \item 2013 - 2017, Ph.D., ETH Zurich
  \begin{itemize}
  \item \textit{Advisor:} Dr. Simon J. Lilly, ETH Zurich
%\item \textit{Thesis:} Evolution of the AGN population in the Universe

\end{itemize}  


%  \item 2012, intership, Jagellonian University
%  \begin{itemize}
%\item \textit{Mentor} dr. sc. Michael Ostrowski, Jagellonian University
%\end{itemize}


    % Received May 16, 2010
  \item 2005 - 2010, MSc, University of Zagreb
  \begin{itemize}
            \item \textit{Advisor:}
      Dr. Hrvoje  Stefancic, Institut Ruder Boskovic
  %  \item  \textit{Thesis:}
  %Unification models of dark energy and dark matter 

\end{itemize}  
\end{itemize}

\section*{Research}
%\subsection*{Research interests}
 Main topics of my work include black hole-galaxy co-evolution, AGN physics and time domain astronomy. %I have developed a phenomenological model which connects black hole evolution with the evolution of galaxies and make numbers of observational predictions, one being that the black hole to galaxy mass ratio must be evolving with redshift. %This projects involved analytical creation of the model, simulating the results and comparison with data. 
 %I have also measured and characterized the optical variability properties of AGNs in the Palomar Transient Factory Survey. I have found that AGN variability often exhibits more correlated behaviour then commonly used random-walk model. %This project involved development of calibration procedures and reduction of the big dataset. 
 
 %In future work I am interested in exploring the connection between the variability approach and the demographical approach to describe the ANG population. I am also interested in researching influence of the AGN spin on the variability and the growth of the accreting black holes.

\subsection*{Peer-Reviewed Journal Articles in Astronomy}

\begin{enumerate}

\item 2018, \textbf{N. Caplar}, S. Lilly, B. Trakhtenbrot\\
AGN evolution from galaxy evolution viewpoint - II, accepted to APJ

\item 2018, L. Sartori, K. Schawinski, B. Trakhtenbrot, \textbf{N. Caplar}, E. Treister, M. Koss, M. Urry, C. Zhang \\
A model for AGN variability on multiple time-scales, 2018, MNRAS, 476L, 34S

\item 2017, A. Weigel, K. Schawinski,  \textbf{N. Caplar}, A. Carpineti, R. Hart, S. Kaviraj, W. Keel, S. Kruk, C. Lintott, R. Nichol, B. Simmons, R. Smethurst
\\    Galaxy Zoo:  Major galaxy mergers are not a significant quenching pathway, APJ, 2017, 845, 145W \\

\item 2017, A. Weigel, K. Schawinski,  \textbf{N. Caplar}, O. I. Wong, T. Ezequiel, B. Trakhtenbrot
\\    Two mass independent Eddington ratio distribution functions regulate black hole growth of blue and red 
galaxies in the local Universe, ApJ, 2017, 845, 134W \\

\item 2017,  \textbf{N. Caplar}, S. J. Lilly, B. Trakhtenbrot   \\Optical variability of AGN in the PTF/iPTF survey, ApJ, 2017, 834, 111C   \\
 
\item 2015,  \textbf{N. Caplar}, S. J. Lilly, B. Trakhtenbrot  \\AGN evolution from a galaxy evolution viewpoint, ApJ,  2015, 811, 148C \\
  
 \item 2013,  \textbf{N. Caplar}, H. Stefancic  \\ Generalized models of unification of dark matter and dark energy, Phys. Rev. D, 2013, 87, 023510 
\end{enumerate}

\newpage 
\subsection*{Other Publications}

These publications are not directly connected to Astronomy. However, they are both ``big-data" papers for which I collected and reduced the data.

\begin{enumerate}[resume]
\item 2016, \textbf{N. Caplar}, S. Tacchella, S. Birrer \\ Quantitative evaluation of gender bias in astronomy,  2017, NatAs, 1E, 182C

\item 2013,  \textbf{N. Caplar}, M. Suznjevic, M. Matijasevic  \\ Analysis of players' in-game performance vs rating: Case study of Heroes of Newerth, Foundation of Digital games 2013,  pp. 237-244
\end{enumerate}

\subsection*{Telescope Proposals}
\begin{itemize}
\item 2013, F. Miniati, S. J. Lilly, \textbf{N. Caplar} \\ The connection between magnetised galactic outflows and high Faraday effect in the circumgalactic environment of intermediate redshift galaxies;  Awarded 24 hours with VIMOS instrument on VLT
\item 2013, S. J. Lilly, F. Miniati,  \textbf{N. Caplar}, B. Gaensler, J. Farnes  \\ Testing the association of magnetized plasma with high redshift galaxies along the line of sight; 
Awarded 5 nights at NTT telescope
\end{itemize}

\section*{Seminar and Conference Presentations}

\begin{itemize}
\item 2018:  New Directions in Optical/Near-IR Spectrographs and Wide-field Imagers, Princeton, USA (conference, talk)

\item 2017:  Weizmann Institute of Science/ University of Geneve/ Unveiling the Physics Behind Extreme AGN Variability, St. Thomas, USA (conference, talk)/ Models of Gravity workshop, Hannover, Germany (workshop, talk)
\item 2016: Caltech/ University of Washington/ Stanford/ University of Maryland/ Shining from the heart of darkness: black hole accretion and jets, Katmandu, Nepal (conference, talk)/ AGN: what's in a name, Munich, Germany (conference, talk)
\item 2015:
Black Hole Accretion and AGN Feedback, Shanghai, PRC (conference, talk)/ Inaugural Zwicky Symposium, Braunwald, Switzerland (conference, talk-organizer)/ Demographics and environment of AGN from multi-wavelength surveys Chania, Greece (conference, talk)/ Unveiling the AGN-Galaxy Evolution, Puerto Varas, Chile (conference, talk)
\item 2014:
COSMOS team meeting, Zagreb, Croatia (workshop, talk)/ Multiwavelength-surveys, Dubrovnik, Croatia (conference, poster)/ Powerful AGN, Port Douglas, Australia (conference, talk)/ The Formation and
Growth of Galaxies in the Young Universe, Obergurgl, Austria (conference, talk)
\item 2012:
Karl-Franzens University/ Jagellonian University/
European Summer Campus, Strasbourg, France (conference-school, poster)
\end{itemize}


%\section*{Teaching}
%As a Ph.D. student at ETH Zurich I had the opportunity to contribute to the teaching
%efforts at the Department of Physics. I committed most of that time helping with the introductory class in physics for chemists at ETH Zurich with $\sim$ 250 students. In the academic year 2015/2016 I led the team which was in charge of preparing homework assignments and preparing the examination questions for this course.

%\begin{itemize}
%\item Assistant: Advanced physics lab, Master course in physics, First year, ETH Zurich: Fall 2016
%\item Assistant: Physics for Chemists, Bachelor course in chemistry, Second year, ETH Zurich: Spring 2016, Fall 2015, Spring 2015, Fall 2014, Spring 2013, Fall 2013
%\item Assistant: Physical Cosmology, Master course in physics, First year, University of Zagreb: Spring 2011 
%\end{itemize}

\section*{Other relevant information}

\begin{itemize}
\item Reviewer for MNRAS and  ApJ
\item Programming Languages: Mathematica, Python, \LaTeX, CIAO, Zemax
\item Experience in working with X-ray, optical and time-domain data
 \item Experience in data reduction, survey calibration, ``big data" and machine learning techniques
\item Since April 2015 I run astrodataiscool.com website, where I publish analysis of the data from \hbox{astronomical} and popular sources. The website gathered $\sim$20000 unique views. 
%\item \textit{Languages:}  \begin{itemize}
%\item Croatian: Native reading, writing, and speaking
%\item English: Fluent reading, writing, and speaking
%\item German : Basic reading, writing, and speaking
%\end{itemize}
\end{itemize}

%\begin{center}

%{\small Last updated: \today}
%\end{center}

\end{document}
\section*{Internships and volunteering}
\begin{itemize}
  \item 2012, Jagellonian University
  \begin{itemize}
     
  
\item \textit{Supervisor:} Dr. Michael Ostrowski
 \item  \textit{Topic:} X-ray variability of knots in the jet of Centaurus A
\end{itemize}  
  \item 2010-2012, Institut Ruder Boskovic
  \begin{itemize}
\item \textit{Supervisor:} Dr. Hrvoje Stefancic
 \item  \textit{Topic:} Generalized models of unification of dark matter and dark energy
\end{itemize}  
\end{itemize}


\section*{References }

\begin{minipage}[t]{0.55\textwidth}
\begin{itemize}
\item \textbf{ dr. Simon J. Lilly, ETH Zurich}
\begin{itemize} \item simon.lilly@phys.ethz.ch
\item  Ph.D. thesis supervisor
\item 
\end{itemize}
\item \textbf{dr. Kevin Schawinski, ETH Zurich}
\begin{itemize} \item  kevin.schawinski@phys.ethz.ch
\item  
\item
\end{itemize}
\item  \textbf{dr. Dave Alexander, Durham University}
\begin{itemize} \item d.m.alexander@durham.ac.uk
\item  
\item 
\end{itemize}
\item \textbf{dr. Hrvoje Stefancic, Institut Ruder Boskovic}
\begin{itemize} \item  shrvoje@thphys.irb.hr
\item  Master thesis supervisor
\item  
\end{itemize}
\end{itemize}
\end{minipage}
\begin{minipage}[t]{0.45\textwidth}
\begin{itemize}
\item
\begin{itemize} 
\item Wolfgang-Pauli-Str. 27
\item 8093 Zurich-Switzerland
\item +41 44 633 38 28 
\end{itemize}
\vspace{0.7 cm}
\begin{itemize} 
\item Wolfgang-Pauli-Str. 27
\item 8093 Zurich-Switzerland
\item +41 44 633 07 51
\end{itemize}
\vspace{0.32 cm}
\begin{itemize} 
\item South Road, 335, Rochester Building
\item  DH1 Durham-United Kingdom
\item +44 191 3343594
\end{itemize}
\vspace{0.35 cm}
\begin{itemize} 
\item Bijenicka cesta 54
\item 10000 Zagreb-Croatia
\item +385 1 456 1032
\end{itemize}
\end{itemize}
\end{minipage}

% Footer
%\bigskip


\end{center}



