\documentclass[11pt,letterpaper]{article}

\usepackage{hyperref}
\usepackage{geometry}
\usepackage{enumitem}

\usepackage[T1]{fontenc}

\def\name{\textbf{Neven Caplar}}

\hypersetup{
  colorlinks = true,
  urlcolor = black,
  pdfauthor = {\name},
  pdfkeywords = {black hole-galaxy co-evolution, time-domain astronomy, AGN, CV},
  pdftitle = {\name: Resume},
  pdfsubject = {Resume},
  pdfpagemode = UseNone
}

\geometry{
  body={6.9in, 9.8in},
  left=0.7in,
  top=0.7in
}

\pagestyle{myheadings}
\markright{\name}
\thispagestyle{empty}

\usepackage{sectsty}
\sectionfont{\rmfamily\bfseries\LARGE}
\subsectionfont{\rmfamily\bfseries\large}

\usepackage{paralist}
  \let\itemize\compactitem
  \let\enditemize\endcompactitem
  \let\enumerate\compactenum
  \let\endenumerate\endcompactenum
  \let\description\compactdesc
  \let\enddescription\endcompactdesc
  \pltopsep=\medskipamount
  \plitemsep=1pt
  \plparsep=1pt

\setlength\parindent{0em}

\renewenvironment{itemize}{
  \begin{list}{}{
    \setlength{\leftmargin}{1.5em}
    \setlength{\itemsep}{0.10em}
    \setlength{\parskip}{0pt}
    \setlength{\parsep}{0.10em}
  }
}{
  \end{list}
}
\setlist[enumerate]{itemsep=0.0em}
\setlist[enumerate]{parskip=0.0em}

\begin{document}

% Place name at left
{\huge \name}

\bigskip

\begin{minipage}[t]{0.645\textwidth}
  Rubin Observatory / University of Washington\\
  Department of Astronomy \\
  Physics-Astronomy Bldg	 \\
  Seattle, WA 98195-1700
\end{minipage}
\begin{minipage}[t]{0.345\textwidth}
  Web: \href{www.ncaplar.com}{www.ncaplar.com} \\
  Email: \href{ncaplar@uw.edu}{\nolinkurl{ncaplar@uw.edu}} \\
  ORCid: \href{https://orcid.org/0000-0003-3287-5250}{0000-0003-3287-5250} \\
  Github: \href{https://github.com/nevencaplar}{nevencaplar}
\end{minipage}

\section*{Work Experience}
\begin{itemize}
  \item Sep 2022 -- Present, Associate Director for LINCC Commissioning, University of Washington \par
  \begin{itemize}
    \item Lead the LSDB project, enabling scalable cross-matching and analysis of petabyte-scale astronomical datasets.
    \item Coordinate cross-institutional efforts with the Space Telescope Science Institute, IPAC, Canadian Astronomy Data Centre, and CDS Strasbourg to provide spatially sharded, Parquet-based catalogs.
    \item Manage a team of six software engineers and one scientist; responsibilities include strategic planning, sprint leadership, and communication with Principal Investigators.
    \end{itemize}
  \item Sep 2022 -- Present, Project Scientist, Rubin Observatory \par
  \begin{itemize}
	\item Developed scripts and code for assessing the quality of image differencing algorithms and construction of lightcurves.
     \end{itemize}
  \item Sep 2017 - Aug 2022, Associate Professional Specialist, Princeton University \par
    \begin{itemize}
    \item Designed algorithms for two-dimensional point spread function modeling for the Prime Focus Spectrograph (Subaru Telescope).
    \item Mentored undergraduate and graduate students (Caltech, Princeton) on AGN variability projects and instrument calibration challenges.
    \end{itemize}
\end{itemize}

\section*{Education}
\begin{itemize}
  \item Ph.D., Science, Swiss Federal Institute of Technology, Zurich, Switzerland 2017 \par
  \item MSc, Physics, University of Zagreb, Zagreb, Croatia, 2010 \par

\end{itemize}

\section*{Research}
\textbf{Main Interests:} Large datasets analysis, software in astronomy, time-domain, stochastic variability. \par
\textbf{Publications:} 19 papers (7 first-author), 6 conference proceedings  (3 first author), 1451 citations


\section*{Skills and Tools}
\begin{itemize}
\item \textbf{Programming:} Python (scientific stack: JAX, DASK, Astropy), LSST Science Pipelines,  Github
\item \textbf{Expertise:} Scalable data reduction, survey calibration, spectroscopy
    \item \textbf{Other:} Wolfram Mathematica, Zemax OpticStudio
\end{itemize}

%\section*{References}
% Available upon request.

\end{document}